\documentclass[a4paper,11pt]{article}
\usepackage[utf8]{inputenc}
\usepackage[spanish]{babel}
\begin{document}
\title{Producción de Biogás a partir de residuos}
\author{Julia Rivera Jiménez}
\date{20-12-2020}
\maketitle
{\em Resumen:} 
El objetivo de este trabajo es proporcionar una visión general de la producción de biogás a partir de residuos lignocelulósicos ya que la demanda mundial de energía es alta y la mayor parte de esta energía se produce a partir de recursos fósiles. El uso irracional actual de combustibles fósiles y el impacto de los gases de efecto invernadero en el medio ambiente están impulsando la investigación sobre la producción de energía renovable a partir de recursos orgánicos y desechos. Estudios recientes informan que la digestión anaeróbica (DA) es una tecnología alternativa eficiente que combina la producción de biocombustible con la gestión sostenible de residuos.
\\
\\
{\em Palabras clave:}
biogás, residuos lignocelulósicos, energía renovable, digestión anaeróbica (DA), biocombustible, sostenible.
\end{document}
