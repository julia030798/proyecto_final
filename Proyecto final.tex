\documentclass[a4paper,11pt]{article}
\usepackage[utf8]{inputenc}
\usepackage[spanish]{babel}

\begin{document}
\title{Producción de Biogás a partir de residuos}
\author{Julia Rivera Jiménez}
\date{20-12-2020}
\maketitle

{\em Resumen:} 
El objetivo de este trabajo es proporcionar una visión general de la producción de biogás a partir de residuos lignocelulósicos ya que la demanda mundial de energía es alta y la mayor parte de esta energía se produce a partir de recursos fósiles. El uso irracional actual de combustibles fósiles y el impacto de los gases de efecto invernadero en el medio ambiente están impulsando la investigación sobre la producción de energía renovable a partir de recursos orgánicos y desechos. Estudios recientes informan que la digestión anaeróbica (DA) es una tecnología alternativa eficiente que combina la producción de biocombustible con la gestión sostenible de residuos.
\\
\\
{\em Palabras clave:}
biogás, residuos lignocelulósicos, energía renovable, digestión anaeróbica (DA), biocombustible, sostenible.

\section{Introducción}
Los escenarios han demostrado que la demanda de energía aumentará durante este siglo en un factor de dos o tres, como resultado del crecimiento de la población y el consumo de energía per cápita. Al mismo tiempo, las concentraciones de gases de efecto invernadero (GEI) en la atmósfera están aumentando rápidamente, siendo las emisiones de $CO_{2}$ derivadas de combustibles fósiles el contribuyente más importante. Las estadísticas indican que los tipos más comunes de combustibles fósiles utilizados en la actualidad son el petróleo y sus productos, el gas natural y el carbón.
\\
\\
Para minimizar los impactos relacionados con el calentamiento global y el cambio climático, las emisiones de GEI deben reducirse a menos de la mitad de los niveles de emisiones globales de 1990. Otro desafío global importante es la seguridad del suministro de energía, porque la mayoría de las reservas convencionales de petróleo y gas conocidas son concentrado en regiones políticamente inestables. 

\end{document}
